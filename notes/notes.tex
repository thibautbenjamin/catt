\documentclass[a4paper]{article}
\usepackage[utf8]{inputenc}
\usepackage[english]{babel}
\usepackage[T1]{fontenc}
\usepackage{amsmath,amssymb,amsfonts}
\usepackage{mathpartir}
\usepackage{macros}

\title{Notes}

\begin{document}
\maketitle
\section{La structure implémentée}
\subsection{Les règles pour les cohérences}
\[
  \inferrule
  {
    \Delta\vdash C:\TCat\\
    \Gamma\vdashps^C\\
    C:\TCat,\Gamma\vdash A : \Type \ C
   }
  {
    \Delta,\Gamma\vdash\coh:A
  }
\]
lorsque $\FV(A)=\set{C}\cup\FV(\Gamma)$

\[
  \inferrule
  {
    \Delta\vdash C:\TCat\\
    \Gamma\vdashps^C\\
    C:\TCat,\Gamma\vdash t\to^C u : \Type \ C \\
    C:\TCat,\partial^-(\Gamma)\vdash t \\
    C:\TCat,\partial^+(\Gamma)\vdash u
  }
  {
    \Delta,\Gamma\vdash\coh:t\to^C u
  }
\]
lorsque $\FV(t)=\set{C}\cup\FV(\partial^-(\Gamma))$ et $\FV(u)=\set{C}\cup\FV(\partial^+(\Gamma))$

\subsection{Les règles pour les foncteurs}
\[
  \inferrule
  {
    \Delta\vdash F:C\To D\\
    \Delta\vdash x : *_C
   }
  {
    \Delta\vdash\ ap(F,x) : *_D
  }
\]


\[
  \inferrule
  {
    \Delta\vdash F:C\To D\\
    \Gamma\vdashps^C\\
    C:\TCat, D:\TCat, F:C\To D,\Gamma\vdash A : \Type \ D
   }
  {
    \Delta,\Gamma\vdash\fcoh:A
  }
\]
lorsque $\FV(t\to^Du)=\set{C,D,F}\cup\FV(\Gamma)$

\[
  \inferrule
  {
    \Delta\vdash F:C\To D\\
    \Gamma\vdashps^C\\
    \Gamma_F,\Gamma\vdash t\to^D u : \Type \ D \\
    \Gamma_F,\partial^-(\Gamma)\vdash t \\
    \Gamma_F,\partial^+(\Gamma)\vdash u
  }
  {
    \Delta,\Gamma\vdash\fcoh:t\to^D u
  }
\]
avec $\Gamma_F = C:\TCat, D:\TCat, F:C\To D$, et \\
lorsque $\FV(t)=\set{C,D,F}\cup\FV(\partial^-(\Gamma))$ et $\FV(u)=\set{C,D,F}\cup\FV(\partial^+(\Gamma))$

\section{Les règles générales à implémenter}

\subsection{les cohérences partiellement généralisées}
\subsubsection{structure}
Pour tout entier n,

\[
  \inferrule
  { }
  {
   \vdash\TCat\ n : \TCat\ n+1
  }
\]

\[
  \inferrule
  {
    \Gamma\vdash C : \TCat\ n
  }
  {
    \Gamma\vdash \Hom\ C : \TCat\ n+1
  }
\]


\[
  \inferrule
  {  }
  {
    \vdash \Hom\ (\TCat\ n+1) \equiv \TCat\ n+1
  }
\]


\[
  \inferrule
  {
    \Gamma\vdash \Hom\ C
  }
  {
    \Gamma\vdash *_C: \Hom\ C
  }
\]


\[
  \inferrule
  { }
  {
    \vdash *_{\TCat\ n+1} \equiv \TCat\ n
  }
\]

\[
  \inferrule
  {
    \Gamma\vdash C : \TCat\ n\\     
    \Gamma\vdash u : *_C \\
    \Gamma\vdash v : *_C
  }
  {
    \Gamma\vdash *_C\ |\ u\to v: \Hom\ C
  }
\]
\[
  \inferrule
  {
    \Gamma\vdash t\ |\ a \to b : \Hom\ C\\
    \Gamma\vdash u : t\ |\ a \to b\\
    \Gamma\vdash v : t\ |\ a \to b
  }
  {
    \Gamma\vdash t\ |\ a \to b\ |\ u\to v: \Hom\ C
  }
\]

\subsubsection{pasting schemes}
\[
  \inferrule
{
    \Gamma\vdashps^C x : *_C
}
{
    \Gamma\vdashps^C
}
\]

\[
  \inferrule
{
    \Gamma\vdashps^C x : A
}
{
    \Gamma,y : A, f : A\ |\ x \to y, \vdashps^C f : A\ |\ x \to y
}
\]

\[
  \inferrule
{ }
{
    (x : *_C)\vdashps^C (x : *_C)
}
\]

\[
  \inferrule
{
    \Gamma\vdashps^C f : A\ |\ x \to y
}
{
    \Gamma\vdashps^C y : A
}
\]


  
\subsubsection{cohérences}
\[
  \inferrule
{
    \Delta\vdashps^{\TCat\ n+1} \\
    \Delta\vdash C : \TCat\ n\\
    \Delta\vdash D : \TCat\ n\\
    \Gamma\vdashps^C\\
    \Delta,\Gamma\vdash A :\Hom \ D \\
  }
  {
    \Delta,\Gamma\vdash\coh : A
  }
  \]
  Lorsque \FV(\Delta)\cup\FV(\Gamma) = \FV(A)

\[
  \inferrule
  {
    \Delta\vdashps^{\Tcat n+1} \\
    \Delta\vdash C : \TCat\ n\\
    \Delta\vdash D : \TCat\ n\\
    \Gamma\vdashps^C\\
    \Delta,\Gamma\vdash t\ |\ u \to v :  \Hom\ D \\
    \Delta,\partial^-(\Gamma)\vdash u\\
    \Delta,\partial^+(\Gamma)\vdash v
  }
  {
    \Delta,\Gamma\vdash\coh : t\ |\ u \to v
  }
  \]
  Lorsque \FV(\Delta)\cup\FV(\partial^-(\Gamma)) = \FV(u) \text{ et } \FV(\Delta)\cup\FV(\partial^+(\Gamma)) = \FV(v)


  \[
  \inferrule
  {
    \Delta\vdashps^{\TCat\ n+1} \\
    \Delta\vdash C : \TCat\ n\\
    \Delta\vdash D : \TCat\ n\\
    \Gamma\vdashps^C\\
    \Delta,\Gamma\vdash t\ |\ u \to v :  \Hom\ D \\
    \partial^-(\Delta),\Gamma\vdash u\\
    \partial^-(\Delta),\Gamma\vdash v
  }
  {
    \Delta,\Gamma\vdash\coh : t\ |\ u \to v
  }
  \]
  Lorsque \FV(\partial^-(\Delta))\cup\FV(\Gamma) = \FV(u) \text{ et } \FV(\partial^+(\Delta))\cup\FV(\Gamma) = \FV(v)

\[
  \inferrule
  {
    \Delta\vdashps^{\TCat\ n+1} \\
    \Delta\vdash C :\TCat\ n\\
    \Delta\vdash D :\TCat\ n\\
    \Delta\vdash *_D :  \Hom\ D \\
  }
  {
    \Delta,(x : *_C)\vdash\coh : *_D
  }
\]
  Lorsque \FV(\partial^-(\Delta)) = \FV(*_D) \text{ et } \FV(\partial^+(\Delta)) = \FV(*_D)
 
\subsection{les cohérences totalement généralisées}
\subsubsection{structure}
Pour tout entier n,

\[
  \inferrule
  {
    \Gamma\vdash C : *_{\TCat\ n}
  }
  {
     \Gamma\vdash *_C : \Type\ n
  }
\]


\[
  \inferrule
  { }
  {
   \vdash\TCat\ n : *_{\TCat\ n+1}
  }
\]


\[
  \inferrule
 {
    \Gamma\vdash *_C : \Type\ n\\
    \Gamma\vdash u : *_C \\
    \Gamma\vdash v : *_C
  }
  {
    \Gamma\vdash *_C\ |\ u\to^C v : \Type\ n
  }
\]
\[
  \inferrule
  {
    \Gamma\vdash t\ |\ a \to^C b : \Type\ n\\
    \Gamma\vdash u : t\ |\ a \to^C b\\
    \Gamma\vdash v : t\ |\ a \to^C b
  }
  {
    \Gamma\vdash t\ |\ a \to^C b\ |\ u\to^C v : \Type\ n
  }
\]

\subsubsection{les listes de contextes}
\begin{itemize}
\item On note :
  $[\Gamma]^k = [\Gamma_k,\cdots,\Gamma_0]$
  Les listes de contextes à $k$ éléments
\item La concaténation est notée \Delta::[\Gamma]

\item Etant donnée une liste de contextes, on définit un prédicat $\vdashps^{[C]}$ inductivement par : 
\[
  \inferrule
  {
    \Gamma_0\vdashps^C
  }
  {      
    [\Gamma_0]\vdashps^{[C]}
  }
\]

\[
  \inferrule
  {
    \Delta\vdashps^D\\
    \Delta\vdash C\\
    [\Gamma]\vdashps^{[C]}
  }
  {      
    \Delta::[\Gamma]\vdashps^{D::[C]}
  }
  \]

\item \partial_i^\pm([\Gamma]) = [\Gamma_k,\cdots,\Gamma_{k-i+2},\partial^\pm(\Gamma_{k-i+1}),\Gamma_{k-i},\cdots,\Gamma_0]
\end{itemize}

  
\subsubsection{les règles de cohérences}
\[
\inferrule
  {
    [\Gamma]^k\vdashps^{(\TCat\ n+k)::[C]^{k-1}}\\
    [\Gamma]\vdash A : \Type\ n
  }
  {
    [\Gamma]\vdash\coh : A
  }
\]
Lorsque \FV([\Gamma]) = \FV(A)


\[
  \inferrule
  {
    [\Gamma]^k\vdashps^{(\TCat\ n+k)::[C]^{k-1}}\\
    [\Gamma]\vdash t\ |\ u \to v :  \Type\ n \\
    \partial_i^-([\Gamma])\vdash u\\
    \partial_i^+([\Gamma])\vdash v
  }
  {
    [\Gamma]\vdash\coh : t\ |\ u \to v
  }
  \]
  Lorsque \FV(\partial_i^-([\Gamma])) = \FV(u) \text{ et } \FV(\partial_i^+(\Gamma)) = \FV(v)

%%%%%% Quelle règle ajouter pour gérer le cas où le but est un objet?
\[
  \inferrule
  {
    [\Gamma]^k\vdashps^{(\TCat\ n+k)::[C]^{k-1}}\\
    [\Gamma]\vdash *_\_ :  \Type\ n \\
    \partial_{k-1}^+([\Gamma])\vdash *
  }
  {
    \Delta,(x : *_C)\vdash\coh : *_D
  }
\]
Lorsque \FV(\partial^-(\Delta)) = \FV(*_C) \text{ et } \FV(\partial^+(\Delta)) = \FV(*_D)
 

\section{Idées de choses à faire}
\begin{itemize}
\item automatisation (de l'associativité), possibilité de traduire les preuve
  de/vers Globular
\item formalisation de l'implémentation avec arguments implicites
\item comment faire des preuves dans le système (e.g. ajouter de la coinduction,
  etc.). Exemple 0 de preuve : montrer que $coh f$ (= cohérence unaire) est
  équivalent à $f$.
\item comment ajouter des $\Pi$ et $\Sigma$ au système de preuve
\item faire la preuve du lien avec la définition de Grothendieck-Maltsiniotis
\item définition des foncteurs
\item généraliser à la catégorie des catégories
\end{itemize}

\end{document}
